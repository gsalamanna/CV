\documentclass{article}
\usepackage{fncylab}
\usepackage{tabularx}
\usepackage{vita}
\usepackage{ifthen}


%----------------------------------------------------------------------

\newboolean{pickrefereed}
\newboolean{disablerefereed}
\newcommand{       \Refereed}[0]{\setboolean{pickrefereed}{true}}
\newcommand{    \NotRefereed}[0]{\setboolean{pickrefereed}{false}}
\newcommand{\DisableRefereed}[0]{\setboolean{disablerefereed}{true}}
\newcommand{ \EnableRefereed}[0]{\setboolean{disablerefereed}{false}}
\DisableRefereed
\Refereed

\newboolean{flagrefereed}
\newcommand{\FlagRefereed}[0]{\setboolean{flagrefereed}{true}}
\newcommand{\DontFlagRefereed}[0]{\setboolean{flagrefereed}{false}}
\DontFlagRefereed

\newcommand{\RefereedFlag}{$^\dagger$}
\newcommand{\SetRefereedFlag}[1]{\renewcommand{\RefereedFlag}{#1}}

\newboolean{pickmostrelevant}
\newboolean{disablemostrelevant}
\newcommand{       \MostRelevant}[0]{\setboolean{pickmostrelevant}{true}}
\newcommand{    \NotMostRelevant}[0]{\setboolean{pickmostrelevant}{false}}
\newcommand{\DisableMostRelevant}[0]{\setboolean{disablemostrelevant}{true}}
\newcommand{ \EnableMostRelevant}[0]{\setboolean{disablemostrelevant}{false}}
\DisableMostRelevant
\MostRelevant

\newboolean{mytrue}
\setboolean{mytrue}{true}
\newboolean{myfalse}
\setboolean{myfalse}{false}

\newboolean{isrefereed}
\newboolean{ismostrelevant}
\newcommand{\IsRefereed}       [0]{\setboolean{isrefereed}{true}}
\newcommand{\IsNotRefereed}    [0]{\setboolean{isrefereed}{false}}
\newcommand{\IsMostRelevant}   [0]{\setboolean{ismostrelevant}{true}}
\newcommand{\IsNotMostRelevant}[0]{\setboolean{ismostrelevant}{false}}

\newboolean{go}
%\newcommand{\ChooseFlag}[2]{}
\newcommand{\ChooseFlag}[2]{
#1
\ifthenelse{
\(
\boolean{disablerefereed}\OR
\(\boolean{pickrefereed}\AND\boolean{isrefereed}\)\OR
\NOT\(\boolean{pickrefereed}\OR\boolean{isrefereed}\)
\)
\AND
\(
\boolean{disablemostrelevant}\OR
\(\boolean{pickmostrelevant}\AND\boolean{ismostrelevant}\)\OR
\NOT\(\boolean{pickmostrelevant}\OR\boolean{ismostrelevant}\)
\)
}
{\setboolean{go}{true} \renewcommand{\AlexLabelPrefix}{}      }
{\setboolean{go}{false}\renewcommand{\AlexLabelPrefix}{Unused}}

\ifthenelse{\boolean{go}}
 {
  \ifthenelse{\boolean{flagrefereed}\AND\boolean{isrefereed}}
             {\renewcommand{\EnumListFlag}{\RefereedFlag}}
             {\renewcommand{\EnumListFlag}{}}
  \item #2 \label{\AlexLabel}
 }{}
}

%\(\(\NOT\boolean{pickmostrelevant}\)\AND\(\(\boolean{pickrefereed}\AND\(#1=0\)\)\OR\(\(\NOT\boolean{pickrefereed}\)\AND #1=1\)\)\)\OR\(\boolean{pickmostrelevant}\AND#2=0\)

\newcommand{\alexcvreference}[1]{}
\newcommand{\alexcvauthor}[1]{{\bf Authors:} #1}
\newcommand{\alexcvtitle}[1]{{\bf Title:} #1}
\newcommand{\alexcvjournal}[1]{{\bf Journal:} #1}
\newcommand{\alexcvvolume}[1]{{\bf Volume:} #1}
\newcommand{\alexcvpages}[1]{{\bf Pages:} #1}
\newcommand{\alexcvyear}[1]{{\bf Date:} #1}
\newcommand{\alexcvSLACcitation}[1]{}
\newcommand{\alexcveprint}[1]{{\bf Electronic Reference:} #1}
\newcommand{\alexcvcollaboration}[1]{{\bf Collaboration:} #1}
\newcommand{\alexcvcdfnote}[1]{{\bf CDF internal note} #1}
\newcommand{\alexcvatlnote}[1]{{\bf ATLAS internal note} #1}
\newcommand{\alexcvnote}[1]{{\bf Additional Information:} #1}

\newcommand{\alexcvsubmittedto}[1]{{\bf Submitted to:} #1}
\newcommand{\alexcvconfproc}[1]{{\bf Proceedings of:} #1}
\newcommand{\alexcvpublinfo}[1]{{\bf Publication:} #1}




\renewcommand{\alexcvreference}[1]{}
\renewcommand{\alexcvauthor}[1]{#1 }
\renewcommand{\alexcvtitle}[1]{{``\it #1'', }}
\renewcommand{\alexcvjournal}[1]{#1}
\renewcommand{\alexcvvolume}[1]{#1, }
\renewcommand{\alexcvpages}[1]{#1}
\renewcommand{\alexcvyear}[1]{(#1)}
\renewcommand{\alexcvSLACcitation}[1]{}
\renewcommand{\alexcveprint}[1]{$\!\!$, (#1)}
\renewcommand{\alexcvcollaboration}[1]{(#1 collaboration), }
\renewcommand{\alexcvnote}[1]{#1}
\renewcommand{\alexcvcdfnote}[1]{(CDF#1)}
\renewcommand{\alexcvsubmittedto}[1]{Submitted to #1}
\renewcommand{\alexcvconfproc}[1]{Proceedings of #1}
\renewcommand{\alexcvpublinfo}[1]{ Publication #1}

%------------------------------------------------------------------------------------

%
\setlength{\textwidth}{6in}
\setlength{\oddsidemargin}{.25in}
\setlength{\evensidemargin}{.25in}
\setlength{\textheight}{8.37in}
\setlength{\topmargin}{0in}
%\setlength{\headsep}{.4in}
%

\newdatedcategory{Public Talks}
\newdatedcategory{Teaching}
\newdatedcategory{Schools}
\newdatedcategory{Research Grant}
\newdatedcategory{Committees}
\newdatedcategory{Positions}
\newdatedcategory{Scientific Leadership}
\newcategory{Technical Skills}
\newenumcategory{SP}{Selected Publications}
\newenumcategory{IN}{Internal Notes}
\newenumcategory{RP}{Refereed Publications}
\newenumcategory{AP}{Additional Publications}
%\newcategory{Publications}
%
%\raggedright
%
\begin{document}
\name{\vbox{\vspace{1cm}}Giuseppe Salamanna}

\businessaddress{
NIKHEF \\
Nationaal instituut voor subatomaire fysica \\
Science Park 105 \\
1098 XG Amsterdam \\
The Netherlands 
}
\businesscontact{
Office 54-3-035 \\
Mailbox A18500 \\
CERN \\
CH-1211 Geneve 23 \\
Switzerland \\
Voice: +41 765-247035 \\
Email: {\tt Giuseppe.Salamanna@cern.ch} 
} 

\begin{vita}
\section*{Title of the PhD thesis}
                 \large{{\it ``First observation of $B_{s}$ mixing at the CDF II experiment with 
                    a newly developed Opposite Side $b$ flavour tagger using Kaons''}},

                 under the advise of Prof. C. Dionisi and Dr. M. Rescigno,
                 Universit\`a degli Studi di Roma {\em La Sapienza}, Rome, Italy \\ 

\section*{Summary of the PhD thesis}
My thesis describes the first observation of flavour oscillations in the $B_{s}$ meson system at the Collider
Detector at Fermilab (CDF). It shows the development, calibration and performance evaluation of
a new Opposite-side b flavour tagger using K mesons in CDF. It also describes the integration of this tagger 
 with the other CDF flavour taggers and how this provides the necessary statistical significance to observe
$B_{s}$ oscillations. The work is performed using data collected by CDF during the Run II of the Tevatron 
hadron collider running at $\sqrt{s} = 1.96~TeV$ at Fermilab. The measurement of $\Delta m_{s}$ is one 
of the milestones of the physics programme of the Tevatron Run II. 
Along with the precise knowledge of its equivalent in the
$B^{0}_{d}$ sector, $\Delta m_{d}$, this measurement provides a very powerful constraint to the
Cabibbo-Kobayashi-Maskawa (CKM) matrix governing quark coupling in weak
interactions. 
Moreover it is sensitive to New Physics contributions from loop diagrams.
CDF observes the $B_s$ oscillations with a statistical significance $>5~\sigma$ 
and is able to measure their frequency. Several experimental ingredients 
contribute to reach this level of significance, of which one of the most crucial is
flavour tagging. This is the determination of the flavour of
the b quark contained within a $B_s$ at creation, before oscillations occurred. 
The effective number of signal events used in the $\Delta m_{s}$ measurement
scales with the figure-of-merit of the combined flavour tagging used, $\varepsilon D^{2}$. 
For my thesis I worked to enhance $\varepsilon D^{2}$. 
I developed an algorithm that looks at the accompanying $b$ quark in the event and tags its flavour 
from the electric charge of its daughter Kaon in the final state. In the thesis, the experimental 
work necessary to achieve a performing tagger in such high track multiplicity environment is 
described. I use particle identification and the displacement of $b$ Kaons from the Primary
Vertex (PV) to discriminate signal Kaons from background like tracks from the underlying event. 
By looking at the flight direction of the decaying $b$ hadrons in events with reconstructed secondary 
decay vertices, I further improve my tagging purity, achieving a performance of the same 
order of the other flavour taggers. I then combine my tagger with the other
 flavour taggers by using a Neural Network (NN). The NN allows to account for correlations
among taggers arising from looking at the same physics process with different 
experimental objects. In the thesis I prove that the OSKT enhances
the performances of the other tags when correlated and that this provides an   
increase in the tagging performances, quantified in $\approx 20\%$ of the 
tagging effectiveness. By doing so, the desired statistical significance
 is achieved and the last chapter is devoted to describe the $\Delta m_{s}$ measurement.
This shows the signal selection and lifetime measurement, which are the other components
of the $\Delta m_{s}$ measurement. The measured value $17.77 \pm 0.10 (stat) \pm 0.07 (syst)~ps^{-1}$
is presented and the uncertainties reported. Finally, the impact on flavour physics is discussed.  

\section*{Composition of thesis committee}
\begin{itemize}
\item President: Prof. Luca Trentadue (Universit\`a degli Studi di Parma, Parma, Italy)
\item Prof. Biagio Saitta (Universit\`a degli Studi di Cagliari, Cagliari, Italy)
\item Prof. Claudio Luci (Universit\`a degli Studi di Roma {\em La Sapienza}, Rome, Italy)
\item Prof. Clara Matteuzzi (Universit\`a degli Studi di Milano {\em Bicocca}, Milan, Italy)
\end{itemize}



\end{vita}

%-------------------------------------------------------------------------------

%\section*{Research Plans}
%\input{research_interest_atlas.tex}

%\section*{Teaching Activity}

\end{document}
