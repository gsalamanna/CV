\documentclass{article}
\usepackage{fncylab}
\usepackage{tabularx}
\usepackage{vita}
\usepackage{ifthen}


%----------------------------------------------------------------------

\newboolean{pickrefereed}
\newboolean{disablerefereed}
\newcommand{       \Refereed}[0]{\setboolean{pickrefereed}{true}}
\newcommand{    \NotRefereed}[0]{\setboolean{pickrefereed}{false}}
\newcommand{\DisableRefereed}[0]{\setboolean{disablerefereed}{true}}
\newcommand{ \EnableRefereed}[0]{\setboolean{disablerefereed}{false}}
\DisableRefereed
\Refereed

\newboolean{flagrefereed}
\newcommand{\FlagRefereed}[0]{\setboolean{flagrefereed}{true}}
\newcommand{\DontFlagRefereed}[0]{\setboolean{flagrefereed}{false}}
\DontFlagRefereed

\newcommand{\RefereedFlag}{$^\dagger$}
\newcommand{\SetRefereedFlag}[1]{\renewcommand{\RefereedFlag}{#1}}

\newboolean{pickmostrelevant}
\newboolean{disablemostrelevant}
\newcommand{       \MostRelevant}[0]{\setboolean{pickmostrelevant}{true}}
\newcommand{    \NotMostRelevant}[0]{\setboolean{pickmostrelevant}{false}}
\newcommand{\DisableMostRelevant}[0]{\setboolean{disablemostrelevant}{true}}
\newcommand{ \EnableMostRelevant}[0]{\setboolean{disablemostrelevant}{false}}
\DisableMostRelevant
\MostRelevant

\newboolean{mytrue}
\setboolean{mytrue}{true}
\newboolean{myfalse}
\setboolean{myfalse}{false}

\newboolean{isrefereed}
\newboolean{ismostrelevant}
\newcommand{\IsRefereed}       [0]{\setboolean{isrefereed}{true}}
\newcommand{\IsNotRefereed}    [0]{\setboolean{isrefereed}{false}}
\newcommand{\IsMostRelevant}   [0]{\setboolean{ismostrelevant}{true}}
\newcommand{\IsNotMostRelevant}[0]{\setboolean{ismostrelevant}{false}}

\newboolean{go}
%\newcommand{\ChooseFlag}[2]{}
\newcommand{\ChooseFlag}[2]{
#1
\ifthenelse{
\(
\boolean{disablerefereed}\OR
\(\boolean{pickrefereed}\AND\boolean{isrefereed}\)\OR
\NOT\(\boolean{pickrefereed}\OR\boolean{isrefereed}\)
\)
\AND
\(
\boolean{disablemostrelevant}\OR
\(\boolean{pickmostrelevant}\AND\boolean{ismostrelevant}\)\OR
\NOT\(\boolean{pickmostrelevant}\OR\boolean{ismostrelevant}\)
\)
}
{\setboolean{go}{true} \renewcommand{\AlexLabelPrefix}{}      }
{\setboolean{go}{false}\renewcommand{\AlexLabelPrefix}{Unused}}

\ifthenelse{\boolean{go}}
 {
  \ifthenelse{\boolean{flagrefereed}\AND\boolean{isrefereed}}
             {\renewcommand{\EnumListFlag}{\RefereedFlag}}
             {\renewcommand{\EnumListFlag}{}}
  \item #2 \label{\AlexLabel}
 }{}
}

%\(\(\NOT\boolean{pickmostrelevant}\)\AND\(\(\boolean{pickrefereed}\AND\(#1=0\)\)\OR\(\(\NOT\boolean{pickrefereed}\)\AND #1=1\)\)\)\OR\(\boolean{pickmostrelevant}\AND#2=0\)

\newcommand{\alexcvreference}[1]{}
\newcommand{\alexcvauthor}[1]{{\bf Authors:} #1}
\newcommand{\alexcvtitle}[1]{{\bf Title:} #1}
\newcommand{\alexcvjournal}[1]{{\bf Journal:} #1}
\newcommand{\alexcvvolume}[1]{{\bf Volume:} #1}
\newcommand{\alexcvpages}[1]{{\bf Pages:} #1}
\newcommand{\alexcvyear}[1]{{\bf Date:} #1}
\newcommand{\alexcvSLACcitation}[1]{}
\newcommand{\alexcveprint}[1]{{\bf Electronic Reference:} #1}
\newcommand{\alexcvcollaboration}[1]{{\bf Collaboration:} #1}
\newcommand{\alexcvcdfnote}[1]{{\bf CDF internal note} #1}
\newcommand{\alexcvatlnote}[1]{{\bf ATLAS internal note} #1}
\newcommand{\alexcvatlpubnote}[1]{{\bf ATLAS public note} #1}
\newcommand{\alexcvnote}[1]{{\bf Additional Information:} #1}

\newcommand{\alexcvsubmittedto}[1]{{\bf Submitted to:} #1}
\newcommand{\alexcvconfproc}[1]{{\bf Proceedings of:} #1}
\newcommand{\alexcvpublinfo}[1]{{\bf Publication:} #1}




\renewcommand{\alexcvreference}[1]{}
\renewcommand{\alexcvauthor}[1]{#1 }
\renewcommand{\alexcvtitle}[1]{{``\it #1'', }}
\renewcommand{\alexcvjournal}[1]{#1}
\renewcommand{\alexcvvolume}[1]{#1, }
\renewcommand{\alexcvpages}[1]{#1}
\renewcommand{\alexcvyear}[1]{(#1)}
\renewcommand{\alexcvSLACcitation}[1]{}
\renewcommand{\alexcveprint}[1]{$\!\!$, (#1)}
\renewcommand{\alexcvcollaboration}[1]{(#1 collaboration), }
\renewcommand{\alexcvnote}[1]{#1}
\renewcommand{\alexcvcdfnote}[1]{(CDF#1)}
\renewcommand{\alexcvsubmittedto}[1]{Submitted to #1}
\renewcommand{\alexcvconfproc}[1]{Proceedings of #1}
\renewcommand{\alexcvpublinfo}[1]{ Publication #1}

%------------------------------------------------------------------------------------

%
\setlength{\textwidth}{6in}
\setlength{\oddsidemargin}{.25in}
\setlength{\evensidemargin}{.25in}
\setlength{\textheight}{8.37in}
\setlength{\topmargin}{0in}
%\setlength{\headsep}{.4in}
%

\newdatedcategory{Public Talks and Seminars}
\newdatedcategory{Teaching}
\newdatedcategory{Schools}
\newdatedcategory{Research Grant}
\newdatedcategory{Committees}
\newdatedcategory{Positions}
\newdatedcategory{Scientific Leadership}
\newcategory{Technical Skills}
\newenumcategory{SP}{Selected Publications}
\newenumcategory{IN}{Internal and Public Notes}
\newenumcategory{RP}{Refereed Publications}
\newenumcategory{AP}{Additional Publications}
%\newcategory{Publications}
%
%\raggedright
%
\begin{document}
\name{\vbox{\vspace{1cm}}Giuseppe Salamanna}

\businessaddress{
NIKHEF \\
Nationaal instituut voor subatomaire fysica \\
Science Park 105 \\
1098 XG Amsterdam \\
The Netherlands 
}
\businesscontact{
Office 54-3-035 \\
Mailbox A18500 \\
CERN \\
CH-1211 Geneve 23 \\
Switzerland \\
Voice: +41 765-247035 \\
Email: {\tt Giuseppe.Salamanna@cern.ch} 
} 

\begin{vita}

{\it Post-doctoral} staff at NIKHEF, under a {\it VIDI} Research
Grant awarded to Dr.~I.~van~Vulpen by the Netherlands Organisation for
Scientific Research (NWO). Based at CERN, where I work full-time on
the {\it ATLAS} experiment at the Large Hadron Collider (LHC) $p-p$ collider. \\
I am actively involved in three aspects of data preparation and
analysis: measurement of the $t\bar{t}$ production cross-section; muon
reconstruction and identification; Level-1 Muon trigger time
calibration.  \\
In the past I have been active in trigger studies for New Physics scenarios in ATLAS. \\
I have obtained my PhD working on the CDF-II experiment, optimizing the CDF Particle Identification
capabilities and developing a 
 Flavour tagging used for the measurement of the $B_s$ mixing frequency. 

\vbox{\vspace{0.5cm}}

\begin{Degrees}
January 2007 & Ph.D. in Physics,
                 with the following thesis:
                 {\it ``First observation of $B_{s}$ mixing at the CDF II experiment with 
                    a newly developed Opposite Side $b$ flavour tagger using Kaons''},
                 under the advise of Prof. C. Dionisi and Dr. M. Rescigno,
                 Universit\`a degli Studi di Roma {\em La Sapienza}, Rome, Italy \\ \\

September 2003 & B.S. and M.S. (``Laurea in Fisica'') with top marks (110/110),
                 with the following thesis:
                 {\it ``Study of the resolution of the Time-Of-Flight detector for the Fermilab CDF experiment''},  
                 under the advise of Prof. C. Dionisi and Dr. S. Giagu, 
                 Universit\`a degli Studi di Roma {\em La Sapienza}, Rome, Italy \\ \\

July 1998 & High School Diploma (``Maturit\`a Classica'') with top marks (60/60),\\
          & Liceo Classico Pilo Albertelli, Rome, Italy\\
\end{Degrees}



\newpage
\begin{Positions}
2008 -- now             & Post-doctoral staff at NIKHEF, based at CERN \\
2008 -- now             & Member of the ATLAS collaboration as NIKHEF staff \\
2007                                & Research Associate position at the University of Washington, based at CERN \\
2006                                & Visiting scientist at the Lawrence Berkeley National Laboratory to collaborate with the local CDF group on the $B_{s}$ mixing analyses  \\
2002 -- 2006               & Visiting scientist at the Fermi National Accelerator Laboratory for the CDF experiment\\
2003 -- 2006               & PhD student, with the CDF group at Universit\`a degli Studi di Roma {\em La Sapienza}\\
2002 -- 2006               & Member of CDF Run II collaboration as an affiliate of University  of Rome\\
2003 -- 2006               & INFN Associate Fellow (``Associazione'') for the CDF experiment \\
2004 -- 2005                  & Teaching assistant (``Aiuto alla didattica'') in Classical Physics (Classical mechanics, thermodynamics, electromagnetism) at the University of Rome {\em La Sapienza} \\
\end{Positions}

\begin{Research Grant}
2003 -- 2006               & Scholarship accompanying my PhD courses, assigned by the Department of Physics, Universit\`a degli Studi di Roma {\em La Sapienza}, after ranking\\
\end{Research Grant}

\begin{Schools}
2004 & CERN European Summer School, Sant Feliu de Guixols, Spain\\
\end{Schools}

\begin{Teaching}
2010 & Supervision of the Ph.D. student N.Ruckstuhl (NIKHEF) for her studies on the muon momentum scale and resolution using an inclusive muon sample from LHC collisions and with cosmic ray events.\\
2009 & Supervision of E.J.Schioppa (INFN-Roma) as a CERN summer student, for his study concerning the timing of the Level-1 Muon barrel trigger with cosmic rays.\\
2008 & Supervision of the NIKHEF Ph.D. student A.Doxiadis (NIKHEF) for his studies on fake leptons rates to be applied to $t\bar{t}$ analysis.\\
2007 & Supervision of the Univeristy of Washington Ph.D. student D.Ventura for his study on Hidden Valley models.\\
2005 & Supervisor of INFN Roma1 Italian student for his work on flavour tagging using $\Lambda$ baryons, during his summer student program at CDF \\
2004 -- 2005 & Teaching assistant (Classical mechanics, thermodynamics and electromagnetism), Undergraduate courses in Pharmacy, University of Rome ``La Sapienza'', Rome, Italy\\
\end{Teaching}

\newpage
\begin{Public Talks and Seminars}
Sep 2010 & ``ATLAS Electroweak results'' , The XIX International Workshop on High Energy Physics and Quantum Field Theory, Golitsyno, Moscow, Russia \\
Feb 2010 & ``Measurement of the Top quark pair production at ATLAS with the first data from LHC'' CPPM Laboratory Seminar, Centre de Physique de Particules de Marseille, Marseille, France \\   
Ott 2009 & ``Results from the ATLAS Barrel Level-1 Muon Trigger timing studies using combined trigger and offline tracking'' 2009 IEEE Nuclear Science Symposium and Medical Imaging Conference (IEEE NSS MIC 09), Orlando, FL, USA \\
Gen 2009 & ``Early Top physics with ATLAS at the LHC'' Physics@FOM Veldhoven 2009 \\
Jul 2006 & ``Measurement of $B_{s}$ oscillations at CDF'' 7th International Conference on Hyperons, Charm And Beauty Hadrons (BEACH06),
             Lancaster, UK. \\
Apr 2006 & ``Measurement of $B_{s}$ oscillation frequency at CDF''
            Incontri di Fisica delle Alte Energie, Pavia, Italy,\\
Feb 2006 & ``$B_{s}$ and sensitivity to new physics at CDF''
            Third workshop on $b$ physics, Parma, Italy,\\
Jul 2005 & ``Techniques for $B_{s}$ Mixing at CDF'' 
           poster at the Hadron Collider Physics Symposium 2005, Les Diablerets, Switzerland\\
Apr 2005 & ``Opposite side B-flavour tagging using combined TOF and dE/dx particle identification technique''
           American Physics Society April Meeting 2005, Tampa, FL, USA \\
Feb 2006 & ``$b$ flavour tagging with Kaons for B physics at CDF''
           RTN ``The third generation as a probe for new physics'' meeting, Karlsruhe, Germany\\
\end{Public Talks and Seminars}

\begin{Technical Skills}
\item {\bf Programming}: C, C++, {\tt Root} analysis package, ATHENA ATLAS analysis framework.
      Knowledgeable in Linux, Windows both at the user and administrator level
\item {\bf Computer Hardware}: PC including several peripherals and interfaces
\item {\bf Trigger}: calibration of Level-1 trigger systems for High Energy Physics.
\item {\bf HEP Detectors}: operation, calibration
\end{Technical Skills}
\vspace{2cm}
\newpage

\begin{References}
  \vbox{\vspace{5mm}}           \\ 
  Prof. Stanislaus M.C. Bentvelsen  \\
  NIKHEF \\
  Science Park 105 \\
  +31 (20) 592-5140 Voice \\
  1098 XG Amsterdam,  The Netherlands \\
  {\tt stanb@nikhef.nl} \\
  \vbox{\vspace{5mm}}                             \\                                                
  Dr. Richard Hawkings \\
  CERN, European Organization for Nuclear Research \\
  Geneve 23, CH-1211, Switzerland \\
  +41 (22) 767 78432 Voice      \\
  {\tt richard.hawkings@cern.ch}       \\
  \vbox{\vspace{5mm}}           \\ 
  Dr. Ludovico Pontecorvo \\
  Istituto Nazionale di Fisica Nucleare - Sezione di Roma, and \\
  CERN, European Organization for Nuclear Research \\
  Geneve 23, CH-1211, Switzerland \\
  +41 (22) 767 78432 Voice      \\
  {\tt ludovico.pontecorvo@cern.ch}       \\
\end{References}
% \vbox{\vspace{5cm}}

\section*{Current and past research Activities}
\setcounter{page}{1}
\subsection*{Current activities in ATLAS}
\subsubsection*{Top physics analysis} 
I am active in the study of $t\bar{t}$ events with high-$P_{T}$
leptons in the final state, selected by the {\it ATLAS} trigger.  The
goal is to measure the $t\bar{t}$ pair production cross-section at the
centre-of-mass energy at which $p-p$ collisions will happen at the
LHC; and to compare the result with the theoretical predictions
provided by the Standard Model (SM). \\
In particular I am looking at semileptonic $t\bar{t}$ decays, with one
lepton, Missing Transverse Energy and high-$P_{T}$ jets in the final state. \\
I am directly involved in improving the analysis at different levels. First of all I am the main 
editor of the internal ATLAS note on Lepton selection and efficiency measurement 
in support of the first cross-section measurement with early LHC data. The note (Ref.\ref{AtlNotes:12}) covers all the relevant information concerning electrons and muons and provides muon efficiency and momentum scale factors to re-scale the simulation to the signal acceptance on data: it is one of the documentation notes for the first $t\bar{t}$ cross-section measurement, in preparation.\\
I am also the coordinator of the sub-group of the ATLAS Top Working Group
that studies and optimizes the muon selections at the analysis level. This group
includes about 15 people from different institutions, meeting on a regular basis to discuss 
muon-related studies in order to 
to quantitatively assess efficiency of muons coming from $t \rightarrow W 
\rightarrow \mu + X$ decays, rejecting secondary muons
coming from quark decays, such as heavy flavours. Problematics like muon trigger signatures for Top physics; muon identification efficiency 
determination from Tag-and-probe techniques using Z bosons; muon fake rate directly from data, are also faced in the group.
The definition and optimization of variables like calorimeter and track-based isolation
and the final performance of the selections is documented in
Ref.~\ref{AtlNotes:1}. These selections are used to perform the first search for top pair candidate events in ATLAS
 at $\sqrt{s} = 7~TeV$ in the single- and di-lepton channels, presented at the ICHEP2010 conference (Refs.\ref{AtlNotes:9},\ref{AtlNotes:10}).  
I was also the Top Working group reference for muons in the analysis readiness review, carried out in December
2009. \\
I have also been collaborating with A.~Doxiadis, student at NIKHEF, in his study, 
fully relying on Monte Carlo information, which establishes a method which links secondary muons and jets, 
making it straightforward to extrapolate a prediction on the fake rate to events with different jet multiplicities. 
His work is in collaboration with colleagues from the French Laboratory LPNHE (``Monte Carlo study of isolated
leptons in multi-jet events'',~Bordoni,~S. {\it et al},~ATL-COM-PHYS-2009-577).\\
I am also working on the final fit to the invariant mass of the
reconstrcted hadronic leg of the
semileptonic $t\bar{t}$ decay, in order to extract the cross-section. This
approach, developed at NIKHEF, has the great advantage to guarantee
that the result is almost 
insensitivite at all to increased W+jets rates with respect to what 
the most widely used Monte-Carlo programs (like Pythia and
McAtNLO) predict. The fit is one of the two mainstream analysis procedures in ATLAS 
for the early cross-section measurement, and results and systematic errors at a centre-of-mass
Energy of 10 TeV have been produced recently (Ref.~\ref{mio:2009xsecnote},~\ref{AtlNotes:2}). In
particular, recently I have been working on how to correctly associate
jets into the reconstructed hadronic top mass, thus reducing the combinatorics
 under the well-reconstructed mass peak. \\
I am also a co-author of the Top {\tt Event Data Model} adopted by the
ATLAS Top WG as a common framework for all Top analysis (Ref.~\ref{AtlNotes:3}).\\

\subsubsection*{Muon reconstruction and analysis of performance}
Concerning muon reconstruction, since more than two years I am one of the developers of the 
{\tt MOORE} program for tracking in the ATLAS Muon Spectrometer, together with a group of colleagues
 mainly from NIKHEF. 
In particular, I have developed a second stage segment finding algorithm, 
which allows to recover hit segments in one ore more layers
of the muon chambers, if they were not associated in the early stages of pattern recognition.
 This tool provides an improvement in resolution for tracks in events with high hit multiplicity in the 
chambers, contributing to improve especially high-$P_{T}$ muon physics in ATLAS. \\
I have performed the analysis of the momentum scale offset and resolution of muons in the ATLAS Muon Spectrometer (MS) relative to the Inner Detector (ID),
on an inclusive muon sample from LHC $p-p$ collisions corresponding to 
$\approx 200~nb^{-1}$. In collaboration with the NIKHEF student Nicole Ruckstuhl, I studied the features of 
the momentum imbalance between the MS and ID measurements on combined muons as a function of kinematics and 
detector characteristics (such as number of MS layers used to build the muon track in the longitudinal and tranverse views). The contribution to the sample due
to a significant fraction of muons coming from pion and kaon decays in flight causes a large tail in the MS-ID momentum imbalance distribution. I estimate this
in the analysis using a template technique, determined from simulation. The final MS-ID momentum imbalance distribution is then fitted with a Gaussian
convoluted with a Landau to describe the heavy flavour component plus the $\pi /K$ template with free normalization. In this way I can obtain a performance estimate 
from the inclusive sample broader in muon $p_{T}$ and more accurate than with the little sample of resonances available from early data. The method is documented in the internal ATLAS note \ref{AtlNotes:14}. The first momentum scale and 
resolution measurement has been published and presented at the ICHEP 2010 conference (Ref.~\ref{AtlNotes:8}). The results are also used 
in the first ATLAS measurement of the $W \rightarrow \mu\nu$ production cross-section and observation of $Z \rightarrow \mu\mu$ production (Ref.~\ref{AtlNotes:11}). \\
I am also one of the software managers of the Muon Calibration framework, which is used at the Calibration
Centers to produce ntuples where all the information regarding hits, segments and tracks in the Muon Spectrometer are 
contained, to calibrate the detector. \\
Finally, in the past I have also been responsible for the validation of the {\tt MOORE}  software chain on simulated samples, 
to monitor performances and stability of the various involved algorithms. and I am an author of the ATLAS document on muon reconstruction performance
 on Monte Carlo events (Ref.~\ref{mio:2009cscmuon}).\\

\subsubsection*{Level-1 Muon trigger timing calibration}
I have as well been involved in the {\it commissioning} 
of the ATLAS Level-1 Muon trigger in the central part of the spectrometer ("Barrel"), provided by 
Resistive Plate Chambers (RPC), using cosmic rays. \\
Having a well timed-in trigger throughout all its geometrical sectors is vital to have a reliable Muon trigger at 
the first stage of the on-line event selection. Infact, if any element is not synchronized with the rest of the 
detector, it will wrong assign the muon to an incorrect LHC bunch crossing, thus biaising the results of the
analysis of that class of events. Therefore I have developed a strategy to time in the various trigger units within 
the same Bunch Crossing interval, corresponding to 25 ns. 
The strategy consists of studying the jitters in the various RPC trigger elements with respect 
to an independent and precise time reference, given by an external trigger source. Within this, I have also proposed use
 of offline tracks, in order to suppress fake triggers. \\ 
This has been tested and applied to 
cosmic ray events, making it possible to time in more than 70$\%$ of the sectors. The aim is now to complete the calibration 
on collision data. For this work I have also developed the tools used in the ATLAS production for the Level-1 Muon barrel trigger
calibration. 
I have also presented these results at the 2009 IEEE NSS-MIC Conference (Ref.~\ref{mio:2009ieee}).

\subsection*{Past analysis activities in ATLAS}
In the year 2007 I have been a Research Associate at the University of Washington (UW), with a post-doctoral position for ATLAS.
In theat period I have been leading the set-up of a study to investigate the detectability of long lived particles that may decay up to several meters away of the primary vertex in the ATLAS detector, and that are a common signature to several non-Standard Model scenarios. The one chosen by us as a reference is Hidden Valley. \\
These signature represent a major challenge already at the trigger level, as this is designed to select objects pointing to the region of primary interaction. The goal in these studies is to propose alternative trigger selections, more efficienct for displaced decays. \\
Together with my UW colleagues, I have proposed new quantities to be used at the Level-2 trigger level, which therefore need to be computed very fastly. The results of the very promising studies are documented, and on the same line the work is now carried on by a group in the ATLAS Exotics Physics Working Group (Ref.~\ref{mio:2009hv},~\ref{AtlNotes:4}). 

\subsection*{Past activities in CDF}
\subsubsection*{$B_{s}$ mixing analysis}
For 4 years, from Summer 2002 to the end of 2006, I have been active in the Collider Detector at Fermilab (CDF) experiment. I have been at the University of Roma ``La Sapienza'' since 1998, when I started my undergraduate studies. In 2003, I received my physics degree under the direction of Professor Carlo Dionisi. In 2007 I obtained my PhD in Physics at the same University and having my thesis work in the CDF experiment. My thesis was defended in January 2007. Since 2002, I have been a visiting scientist at the Fermi National Accelerator Laboratory. My research concerned experimental HEP in the flavour sector. My work consisted of developing a flavour tagger using K mesons produced in $B_{s}$ signal events. I demonstrated for the first time that it is possible to have a non-null tagging performance for such an algorithm at a hadron collider, where a very high track multiplicity is present. My work contributed to the CDF measurement of the $B_{s}^{0}$ mixing frequency, $\Delta M_{s}$, long waited as an important constraint to possible New Physics in the flavour sector; and uniquely possible at CDF (Ref.~\ref{Abulencia:2006obs}).\\
In the summer 2006 I have also actively collaborated with the CDF group at the Lawrence Berkeley National Laboratory, directed by Prof.~M.~D.~Shapiro,  in order to develop an independent measurent of  $\Delta M_{s}$ based on Fourier Tranform: instead of using the mainstream {\it amplitude method}, as a final statistical technique to find the value of the oscillation frequency, we have proposed to directly look at the space of frequencies, to search for a peak in the frequency distribution, as the Transform space of the time domain. I have spent one month in Berkeley, working on the preparation of flavour tagging for the measurement. \\
\newline 
A relevant part of the $\Delta M_{s}$ analysis is to identify the flavour of the b quark in the $B_{s}^{0}$ meson by looking at tracks coming from the same fragmentation process or from decays of the accompanying $b$ quark. In such flavour tagging algorithms the identification of the Kaons through the CDF Time-Of-Flight (TOF) detector, plays a major role, without which CDF couldn't have reached the necessary sensitivity to mixing. 


\subsubsection*{Time of Flight detector calibration}
I have been working on the CDF Time-Of-Flight detector since 2002, focusing on resolution studies and off-line calibrations in order to improve the separation power of the various particle species for Particle Identification. I was also in charge of validating such performances during the various periods of data taking. I have further contributed to CDF Particle Id performances through the study of a combination of the measurements of a particle's Time-Of-Flight and specific ionization in the central drift chamber. I applied this combined technique to develop a flavour tagging algorithm for B physics (in particular $B_{s}$ mixing) at CDF. This is based on the identification of a Kaon coming from a decay chain of the kind $b \rightarrow c \rightarrow s$. The charge of such a displaced kaon is related to the flavour of the away $b$ hadron and is used in the present CDF combined Opposite Side Tagger to enhance our capabilities to tell the mixing $B_{s}$ flavour at creation. This tagger has been combined with the other taggers in place to increase our $\Delta M_{s}$ measurement significance in the sunsequent rounds of analyses that have lead to the first observation of the phenomenon.

\clearpage


\SetAttitionalText{This list contains the publications for which I was a main author. Refereed papers are indicated with a \RefereedFlag.}
\begin{Selected Publications}
\EnableMostRelevant
\MostRelevant
\DisableRefereed
\FlagRefereed
\input{../PUBBLICATION_LIST/publist_giuseppe}
\end{Selected Publications}

\clearpage

\SetAttitionalText{This list covers the internal and public notes of the CDF and ATLAS experiments where I was (co-)author.}
\begin{Internal and Public Notes}
\DisableMostRelevant
\DisableRefereed
%----------------------------------------------------------------------------------------------
\ChooseFlag{\IsNotRefereed\IsMostRelevant}{
\alexcvlabel{AtlNotes:20121}
    \alexcvtitle     {Measurement of the top quark pair production cross section with ATLAS in pp collisions at $\sqrt{s}$ = 7 TeV in the single-lepton channel using semileptonic $b$ decays} 
    \alexcvatlconfnote      {ATL-CONF-2012-131}
}

\ChooseFlag{\IsNotRefereed\IsNotMostRelevant}{
\alexcvlabel{AtlNotes:01}
    \alexcvtitle     {Measurement of the Top Quark Pair Production Cross-section in ATLAS in the Single Lepton plus Jets Channel}
    \alexcvatlnote      {ATL-COM-PHYS-2011-666}
}

\ChooseFlag{\IsNotRefereed\IsMostRelevant}{
\alexcvlabel{AtlNotes:03}
    \alexcvtitle     {Measurement of the top quark-pair cross-section with ATLAS in pp collisions at sqrt(s) = 7 TeV in the single-lepton channel using b-tagging}
    \alexcvatlnote      {ATL-COM-CONF-2011-028}
}


\ChooseFlag{\IsNotRefereed\IsNotMostRelevant}{
\alexcvlabel{AtlNotes:04}
    \alexcvtitle     {Di-muon invariant mass resolution in Z->mu+mu- decays : document supporting the request for approval of performance plots}
    \alexcvatlnote      {ATL-COM-CONF-2010-904}
}

\ChooseFlag{\IsNotRefereed\IsMostRelevant}{
\alexcvlabel{AtlNotes:05}
    \alexcvtitle     {Calibration of the $\chi^{2}_{match}$-based Soft Muon Tagger algorithm}
    \alexcvatlnote      {ATL-COM-PHYS-2012-008}
}

\ChooseFlag{\IsNotRefereed\IsMostRelevant}{
\alexcvlabel{AtlNotes:06}
    \alexcvtitle     {Calibration of the $\chi^{2}_{match}$-based Soft Muon Tagger algorithm using 2012 ATLAS data}
    \alexcvatlnote      {ATL-COM-MUON-2013-031}
}

\ChooseFlag{\IsNotRefereed\IsMostRelevant}{
\alexcvlabel{AtlNotes:07}
    \alexcvtitle     {Object Selections and Background estimates in the $H\rightarrow WW^{(*)}$ analysis with 20.7 $fb^{-1}$ of data collected with the ATLAS detector at $\sqrt{s} = 8$ TeV}
    \alexcvatlnote      {ATL-COM-PHYS-2013-1504}
}

\ChooseFlag{\IsNotRefereed\IsNotMostRelevant}{
\alexcvlabel{AtlNotes:1}
    \alexcvtitle     {Study on reconstructed object definition and selection for top physics}
    \alexcvatlnote      {ATL-COM-PHYS-2009-633}
}

\ChooseFlag{\IsNotRefereed\IsMostRelevant}{
\alexcvlabel{AtlNotes:20101}
    \alexcvtitle     {Performance of the ATLAS Muon Trigger in p-p collisions at $\sqrt{s}$ = 7 TeV}
    \alexcvatlconfnote      {ATL-CONF-2010-095}
}

\ChooseFlag{\IsNotRefereed\IsNotMostRelevant}{
\alexcvlabel{AtlNotes:02}
    \alexcvtitle     {ATLAS muon reconstruction efficiency and dimuon mass resolution in early 2011 LHC collisions data}
    \alexcvatlnote      {ATL-COM-CONF-2011-094}
}


\ChooseFlag{\IsNotRefereed\IsMostRelevant}{
\alexcvlabel{AtlNotes:55}
    \alexcvtitle     {{\bf Main editor:} Object selection and calibration, background estimations and MC samples for the Winter 2012 Top Quark analyses with 2011 data}
    \alexcvatlnote      {ATL-COM-PHYS-2012-224, ATL-COM-PHYS-2012-449 (update for Summer 2012), ATL-COM-PHYS-2013-088 (update for Winter 2013)}
}

\ChooseFlag{\IsNotRefereed\IsNotMostRelevant}{
\alexcvlabel{AtlNotes:2}
   \alexcvtitle     {Prospects for measuring the Top Quark Pair Production Cross-section in the Single Lepton Channel at ATLAS in 10 TeV $p-p$ Collisions}
   \alexcvatlpubnote      {ATL-PHYS-INT-2009-071, ATL-COM-PHYS-2009-306}
}

\ChooseFlag{\IsNotRefereed\IsMostRelevant}{
\alexcvlabel{AtlNotes:4}
    \alexcvtitle     {Detection of long lived neutral particles in the ATLAS detector}
    \alexcvatlpubnote     {ATL-COM-PHYS-2008-020}
}

\ChooseFlag{\IsNotRefereed\IsMostRelevant}{
\alexcvlabel{AtlNotes:8}
    \alexcvtitle     {Muon reconstruction performance}
    \alexcvatlpubnote      {ATLAS-CONF-2010-064}
}

\ChooseFlag{\IsNotRefereed\IsNotMostRelevant}{
\alexcvlabel{AtlNotes:9}
    \alexcvtitle     {Search for top pair candidate events in ATLAS at $\sqrt{s}~=~7~ TeV$}
    \alexcvatlconfnote      {ATLAS-CONF-2010-063}
}

\ChooseFlag{\IsNotRefereed\IsMostRelevant}{
\alexcvlabel{AtlNotes:12}
      \alexcvtitle     {{\bf Main editor:} Lepton Trigger and Identification for the first {\it Top quark} observation
}
\alexcvatlnote      {ATLAS-COM-PHYS-2010-826}
}

\ChooseFlag{\IsNotRefereed\IsNotMostRelevant}{
\alexcvlabel{AtlNotes:14}
      \alexcvtitle     {{\bf Main editor:} Muon momentum scale and resolution measurements with inclusive muons from $1.2~pb^{-1}$ of collision data at $\sqrt{s}$ = 7 TeV
}
\alexcvatlnote      {ATL-COM-PHYS-2010-708}
}

\ChooseFlag{\IsNotRefereed\IsNotMostRelevant}{
\alexcvlabel{AtlNotes:3}
   \alexcvtitle     {{\bf Main editor:} Design Considerations for the Top reconstruction Output EDM Classes}
   \alexcvatlnote      {ATL-COM-SOFT-2009-006}
}

\ChooseFlag{\IsNotRefereed\IsNotMostRelevant}{
\alexcvlabel{AtlNotes:5}
    \alexcvtitle     {Accompanying note for approval of plots from Level 1 Muon Barrel trigger timing studies}
    \alexcvatlnote      {ATL-COM-MUON-2009-034}
}

\ChooseFlag{\IsNotRefereed\IsNotMostRelevant}{
\alexcvlabel{AtlNotes:6}
    \alexcvtitle     {Atlas Muon Trigger Performance on cosmics and p-p collisions at $\sqrt{s}~=~900~GeV$}
    \alexcvatlpubnote    {ATL-COM-DAQ-2010-011}
}

\ChooseFlag{\IsNotRefereed\IsNotMostRelevant}{
\alexcvlabel{AtlNotes:7}
    \alexcvtitle     {Muon Performance in Minimum Bias $pp$ Collision Data at $\sqrt{s}~=~7~TeV$ with ATLAS}
    \alexcvatlconfnote      {ATLAS-CONF-2010-036}
}

\ChooseFlag{\IsNotRefereed\IsNotMostRelevant}{
\alexcvlabel{AtlNotes:10}
    \alexcvtitle     {Expected event distributions for early top pair candidates in ATLAS at $\sqrt{s}~=~7~ TeV$}
    \alexcvatlpubnote      {ATL-PHYS-PUB-2010-012}
}

\ChooseFlag{\IsNotRefereed\IsNotMostRelevant}{
\alexcvlabel{AtlNotes:11}
    \alexcvtitle     {Measurement of the $W \rightarrow \ell\nu$ production cross-section and observation of $Z \rightarrow \ell\ell$ production in proton-proton collisions at $\sqrt{s}~=~7~TeV$ with the ATLAS detector}
    \alexcvatlconfnote      {ATL-CONF-2010-051}
}

\ChooseFlag{\IsNotRefereed\IsMostRelevant}{
\alexcvlabel{AtlNotes:13}
      \alexcvtitle     {Background studies for top-pair production in lepton plus jets final states in $\sqrt{s}~=~7~TeV$ ATLAS data
}
\alexcvatlnote      {ATLAS-COM-CONF-2010-085}

}

\ChooseFlag{\IsNotRefereed\IsMostRelevant}{
\alexcvlabel{AtlNotes:14}
      \alexcvtitle     {{\bf Main editor:} Muon Momentum Resolution in First Pass Reconstruction of pp Collision Data Recorded by ATLAS in 2010
}
\alexcvatlconfnote      {ATLAS-CONF-2011-046}
}

\ChooseFlag{\IsNotRefereed\IsMostRelevant}{
\alexcvlabel{AtlNotes:15}
      \alexcvtitle     {Measurement of the top quark-pair cross-section with ATLAS in pp collisions at $\sqrt{s}~=~7~TeV$ in the single-lepton channel using b-tagging}
\alexcvatlconfnote      {ATLAS-CONF-2011-035}
}

\ChooseFlag{\IsNotRefereed\IsMostRelevant}{
\alexcvlabel{AtlNotes:16}
      \alexcvtitle     {{\bf Main editor:} Lepton trigger and identification for the Winter 2011 top quark analyses}
\alexcvatlnote      {ATL-COM-PHYS-2011-123}
}

\ChooseFlag{\IsNotRefereed\IsNotMostRelevant}{
\alexcvlabel{AtlNotes:17}
      \alexcvtitle     {Measurement of the top quark cross-section in the semileptonic channel at $sqrt{s}=7TeV$ with the ATLAS detector}
\alexcvatlnote      {ATL-COM-PHYS-2011-111}
}

\ChooseFlag{\IsNotRefereed\IsNotMostRelevant}{
\alexcvlabel{CDFNotes:1}
     \alexcvtitle     {Determination of $B^{0}$ and $B^{+}$ Lifetimes
     in Hadronic Decays Using Partially and Fully Reconstructed Modes
     without Event-by-Event $ct$ Resolutions}
     \alexcvcdfnote      {9139}  
}

\ChooseFlag{\IsNotRefereed\IsNotMostRelevant}{
\alexcvlabel{CDFNotes:2}
     \alexcvtitle     {First observation of $\bar{B}_{s}^{0} \rightarrow
    D_{s}^{\pm}K^{\mp}$ and measurement of the relative branching
     fraction BR($\bar{B}_{s}^{0} \rightarrow
     D_{s}^{\pm}K^{\mp}$)/BR($\bar{B}_{s}^{0} \rightarrow
     D_{s}^{+}\pi^{-}$)}
    \alexcvcdfnote      {8850}
}
\ChooseFlag{\IsNotRefereed\IsNotMostRelevant}{
\alexcvlabel{CDFNotes:3}
     \alexcvtitle     {Determination of $B^{0}$ and $B^{+}$ Lifetimes
     in Hadronic Decays Using Partially and Fully Reconstructed Modes}
     \alexcvcdfnote      {8778}
}
\ChooseFlag{\IsNotRefereed\IsNotMostRelevant}{
\alexcvlabel{CDFNotes:4}
     \alexcvtitle     {Measurement of $B^{-}$ Relative Branching
     Fractions with a Combined Mass and $dE/dx$ Fit}
     \alexcvcdfnote      {8777}
}
\ChooseFlag{\IsNotRefereed\IsNotMostRelevant}{
\alexcvlabel{CDFNotes:5}
     \alexcvtitle     {Measurement of $B_{s}^{0}$ Branching
     Fractions Using Combined Mass and $dE/dx$ Fits}
     \alexcvcdfnote      {8776}
}
\ChooseFlag{\IsNotRefereed\IsNotMostRelevant}{
\alexcvlabel{CDFNotes:6}
     \alexcvtitle     {Measurement of $B^{0}$ Branching
     Fractions Using Combined Mass and $dE/dx$ Fits}
     \alexcvcdfnote      {8705}
}
\ChooseFlag{\IsNotRefereed\IsNotMostRelevant}{
\alexcvlabel{CDFNotes:7}
     \alexcvtitle     {$B_{s} \rightarrow
     D_{s}\pi$ with $D_{s} \rightarrow
     K_{s}[\pi\pi]K$ channel reconstruction using PID}
     \alexcvcdfnote      {8520}
}

\ChooseFlag{\IsNotRefereed\IsMostRelevant}{
\alexcvlabel{CDFNotes:8}
     \alexcvtitle     {Combined opposite side flavor tagger}
     \alexcvcdfnote      {8314}
}

\ChooseFlag{\IsNotRefereed\IsMostRelevant}{
\alexcvlabel{CDFNotes:9}
     \alexcvtitle     {Opposite Side Kaon Tagging}
     \alexcvcdfnote      {8179}
}
\ChooseFlag{\IsNotRefereed\IsMostRelevant}{
\alexcvlabel{CDFNotes:10}
     \alexcvtitle     {dE/dx, TOF validation studies on 0h/0i data for Bs mixing analyses}
     \alexcvcdfnote      {8169}
}

\ChooseFlag{\IsNotRefereed\IsMostRelevant}{
\alexcvlabel{CDFNotes:18}
     \alexcvtitle     {Tof Resolution studies using muons from $J/\psi$}
     \alexcvcdfnote      {6810}
}

\ChooseFlag{\IsNotRefereed\IsNotMostRelevant}{
\alexcvlabel{CDFNotes:11}
     \alexcvtitle     {Improving the dE/dx modeling and a study of the composition of the $B \rightarrow hh$ background}
     \alexcvcdfnote      {7646}
}
\ChooseFlag{\IsNotRefereed\IsMostRelevant}{
\alexcvlabel{CDFNotes:12}
     \alexcvtitle     {Particle Identification by combining TOF and dE/dx information}
     \alexcvcdfnote      {7488}
}
\ChooseFlag{\IsNotRefereed\IsNotMostRelevant}{
\alexcvlabel{CDFNotes:13}
     \alexcvtitle     {$B\_PIPI$ trigger at high luminosity}
     \alexcvcdfnote      {7320}
}
\ChooseFlag{\IsNotRefereed\IsNotMostRelevant}{
\alexcvlabel{CDFNotes:14}
     \alexcvtitle     {Measurement of isolation efficiency in low pt B mesons}
     \alexcvcdfnote      {7066}
}
\ChooseFlag{\IsNotRefereed\IsNotMostRelevant}{
\alexcvlabel{CDFNotes:15}
     \alexcvtitle     {Branching Ratios and CP asymmetries in $B \rightarrow hh$ decays from 180 $pb^{-1}$}
     \alexcvcdfnote      {7049}
}

\ChooseFlag{\IsNotRefereed\IsNotMostRelevant}{
\alexcvlabel{CDFNotes:16}
     \alexcvtitle     {Upper limit on $\Lambda_b$ to hh}
     \alexcvcdfnote      {7048}
}
\ChooseFlag{\IsNotRefereed\IsNotMostRelevant}{
\alexcvlabel{CDFNotes:17}
     \alexcvtitle     {Search for $B_{s} \rightarrow \phi \phi$ decays at CDF}
     \alexcvcdfnote      {6937}
}


\end{Internal and Public Notes}

\clearpage

\SetAttitionalText{This list covers the refereed articles co-authored by me.}
\begin{Refereed Publications}
% Pick only Refereed papers
\EnableRefereed
\Refereed
% without flagging them
%\DontFlagRefereed
% and removing objects from the ``most relevant'' list
\EnableMostRelevant
\NotMostRelevant
\input{../PUBBLICATION_LIST/publist_giuseppe}
\end{Refereed Publications}

%\vbox{\vspace{20cm}}
\clearpage

\end{vita}

%-------------------------------------------------------------------------------

%\section*{Research Plans}
%\input{research_interest_atlas.tex}

%\section*{Teaching Activity}

\end{document}
