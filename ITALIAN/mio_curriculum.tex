\documentclass{article}
\usepackage{fncylab}
\usepackage{tabularx}
\usepackage{vita}
\usepackage{ifthen}


%----------------------------------------------------------------------

\newboolean{pickrefereed}
\newboolean{disablerefereed}
\newcommand{       \Refereed}[0]{\setboolean{pickrefereed}{true}}
\newcommand{    \NotRefereed}[0]{\setboolean{pickrefereed}{false}}
\newcommand{\DisableRefereed}[0]{\setboolean{disablerefereed}{true}}
\newcommand{ \EnableRefereed}[0]{\setboolean{disablerefereed}{false}}
\DisableRefereed
\Refereed

\newboolean{flagrefereed}
\newcommand{\FlagRefereed}[0]{\setboolean{flagrefereed}{true}}
\newcommand{\DontFlagRefereed}[0]{\setboolean{flagrefereed}{false}}
\DontFlagRefereed

\newcommand{\RefereedFlag}{$^\dagger$}
\newcommand{\SetRefereedFlag}[1]{\renewcommand{\RefereedFlag}{#1}}

\newboolean{pickmostrelevant}
\newboolean{disablemostrelevant}
\newcommand{       \MostRelevant}[0]{\setboolean{pickmostrelevant}{true}}
\newcommand{    \NotMostRelevant}[0]{\setboolean{pickmostrelevant}{false}}
\newcommand{\DisableMostRelevant}[0]{\setboolean{disablemostrelevant}{true}}
\newcommand{ \EnableMostRelevant}[0]{\setboolean{disablemostrelevant}{false}}
\DisableMostRelevant
\MostRelevant

\newboolean{mytrue}
\setboolean{mytrue}{true}
\newboolean{myfalse}
\setboolean{myfalse}{false}

\newboolean{isrefereed}
\newboolean{ismostrelevant}
\newcommand{\IsRefereed}       [0]{\setboolean{isrefereed}{true}}
\newcommand{\IsNotRefereed}    [0]{\setboolean{isrefereed}{false}}
\newcommand{\IsMostRelevant}   [0]{\setboolean{ismostrelevant}{true}}
\newcommand{\IsNotMostRelevant}[0]{\setboolean{ismostrelevant}{false}}

\newboolean{go}
%\newcommand{\ChooseFlag}[2]{}
\newcommand{\ChooseFlag}[2]{
#1
\ifthenelse{
\(
\boolean{disablerefereed}\OR
\(\boolean{pickrefereed}\AND\boolean{isrefereed}\)\OR
\NOT\(\boolean{pickrefereed}\OR\boolean{isrefereed}\)
\)
\AND
\(
\boolean{disablemostrelevant}\OR
\(\boolean{pickmostrelevant}\AND\boolean{ismostrelevant}\)\OR
\NOT\(\boolean{pickmostrelevant}\OR\boolean{ismostrelevant}\)
\)
}
{\setboolean{go}{true} \renewcommand{\AlexLabelPrefix}{}      }
{\setboolean{go}{false}\renewcommand{\AlexLabelPrefix}{Unused}}

\ifthenelse{\boolean{go}}
 {
  \ifthenelse{\boolean{flagrefereed}\AND\boolean{isrefereed}}
             {\renewcommand{\EnumListFlag}{\RefereedFlag}}
             {\renewcommand{\EnumListFlag}{}}
  \item #2 \label{\AlexLabel}
 }{}
}

%\(\(\NOT\boolean{pickmostrelevant}\)\AND\(\(\boolean{pickrefereed}\AND\(#1=0\)\)\OR\(\(\NOT\boolean{pickrefereed}\)\AND #1=1\)\)\)\OR\(\boolean{pickmostrelevant}\AND#2=0\)

\newcommand{\alexcvreference}[1]{}
\newcommand{\alexcvauthor}[1]{{\bf Authors:} #1}
\newcommand{\alexcvtitle}[1]{{\bf Title:} #1}
\newcommand{\alexcvjournal}[1]{{\bf Journal:} #1}
\newcommand{\alexcvvolume}[1]{{\bf Volume:} #1}
\newcommand{\alexcvpages}[1]{{\bf Pages:} #1}
\newcommand{\alexcvyear}[1]{{\bf Date:} #1}
\newcommand{\alexcvSLACcitation}[1]{}
\newcommand{\alexcveprint}[1]{{\bf Electronic Reference:} #1}
\newcommand{\alexcvcollaboration}[1]{{\bf Collaboration:} #1}
\newcommand{\alexcvcdfnote}[1]{{\bf CDF internal note} #1}
\newcommand{\alexcvatlnote}[1]{{\bf ATLAS internal note} #1}
\newcommand{\alexcvatlpubnote}[1]{{\bf ATLAS public note} #1}
\newcommand{\alexcvnote}[1]{{\bf Additional Information:} #1}

\newcommand{\alexcvsubmittedto}[1]{{\bf Submitted to:} #1}
\newcommand{\alexcvconfproc}[1]{{\bf Proceedings of:} #1}
\newcommand{\alexcvpublinfo}[1]{{\bf Publication:} #1}




\renewcommand{\alexcvreference}[1]{}
\renewcommand{\alexcvauthor}[1]{#1 }
\renewcommand{\alexcvtitle}[1]{{``\it #1'', }}
\renewcommand{\alexcvjournal}[1]{#1}
\renewcommand{\alexcvvolume}[1]{#1, }
\renewcommand{\alexcvpages}[1]{#1}
\renewcommand{\alexcvyear}[1]{(#1)}
\renewcommand{\alexcvSLACcitation}[1]{}
\renewcommand{\alexcveprint}[1]{$\!\!$, (#1)}
\renewcommand{\alexcvcollaboration}[1]{(#1 collaboration), }
\renewcommand{\alexcvnote}[1]{#1}
\renewcommand{\alexcvcdfnote}[1]{(CDF#1)}
\renewcommand{\alexcvatlnote}[1]{(#1)}
\renewcommand{\alexcvsubmittedto}[1]{Submitted to #1}
\renewcommand{\alexcvconfproc}[1]{Proceedings of #1}
\renewcommand{\alexcvpublinfo}[1]{ Publication #1}

%------------------------------------------------------------------------------------

%
\setlength{\textwidth}{6in}
\setlength{\oddsidemargin}{.25in}
\setlength{\evensidemargin}{.25in}
\setlength{\textheight}{8.37in}
\setlength{\topmargin}{0in}
%\setlength{\headsep}{.4in}
%

\newdatedcategory{Istruzione}
\newdatedcategory{Talks a conferenze e Seminari}
\newdatedcategory{Insegnamento}
\newdatedcategory{Scuole}
\newdatedcategory{Borse di studio e ricerca}
\newdatedcategory{Carriera}
\newcategory{Conoscenze tecniche}
\newenumcategory{SP}{Selezione delle Pubblicazioni}
\newenumcategory{IN}{Note interne e pubbliche}
\newenumcategory{RP}{Pubblicazioni sotto referee}

\begin{document}
\name{\vbox{\vspace{1cm}}Giuseppe Salamanna}

\businessaddress{NIKHEF  \\
Nationaal instituut voor subatomaire fysica, \\ 
Science Park 105 \\
1098 XG Amsterdam \\
The Netherlands
}
\businesscontact{
Distaccato al CERN \\
Office 54-3-035, 
Mailbox A18500 \\
CERN \\
CH-1211 Geneve 23 \\
Switzerland \\
Telefono: +41-765-247035 \\
Email: {\tt Giuseppe.Salamanna@cern.ch} 
} 

\begin{vita}
{\it Post-doctoral staff} a NIKHEF, con una borsa di ricerca
{\it VIDI} sovvenzionata dal governo olandese. Distaccato al CERN, dove lavora a tempo pieno nell'esperimento ATLAS. Le sue attuali
attivit\`{a} di ricerca includono il {\it commissioning} del Livello-1 del trigger dei muoni di ATLAS, il lavoro di sviluppo ed ottimizzazione
della ricostruzione di muoni; e l'analisi di fisica del quark {\it top}
in vista dei primi dati dell'acceleratore $p-p$ LHC. \\
Da diversi mesi \`e coinvolto, per una parte consistente del tempo, 
nel {\it commissioning} del sistema di trigger di muoni di Livello-1 di ATLAS, utilizzando
raggi cosmici. Si sta occupando di valutare il grado
di sincronizzazione dei vari settori di trigger utilizzando in parallelo due strategie: studiare i ritardi in
riferimento ad un tempo dato da un trigger esterno; e guardare esclusivamente allo stesso trigger di muoni, in relazione 
ad un settore scelto come riferimento; questa ultima analisi massimizza la accettanza ai raggi cosmici.  
Ha proposto, per questa analisi, l'utilizzo di tracce di muoni ricostruite, con il molteplice vantaggio di 
sopprimere falsi trigger, e poter selezionare differenti topologie per lo studio (Ref.~\Ref{mio:2009ieee}). \\
Ha anche effettuato uno studio preliminare delle prestazioni del trigger di muoni di Livello-1 con i primi dati a 900 GeV 
di LHC del Dicembre 2009 (Ref.~\Ref{AtlNotes:6}).
Per il lavoro di commissioning ha prodotto una notevole quantit\`{a} di strumenti di analisi
 e validazione, utilizzati nella {\it production} ufficiale di ATLAS; 
per passare poi alla analisi vera e propria. 
 Il lavoro \`e in stretta collaborazione con gli
esperti del trigger di muoni (in gran parte italiani).  \\
\newline
Per quanto concerne la ricostruzione dei muoni, \`e da pi\`u di due
anni uno degli sviluppatori del programma {\tt Moore} di tracciamento nello
spettrometro di muoni di ATLAS. In particolare, ha sviluppato il
secondo stadio del {\it segment finding} di Moore, in cui vengono
recuperati i segmenti di traccia non
associati ad un muone nei primi stadi della ricostruzione. Tale
recupero constribuisce notevolmente a migliorare la risoluzione in
impulso per tracce in eventi ad alta molteplicit\`a di {\it hit} nelle
camere a mu. \\ 
Al fine di controllare le prestazioni di Moore \`e anche il
responsabile della validazione della intera catena di ricostruzione su
campioni di eventi simulati. Infine, \`e anche tra gli autori
dei risultati di ATLAS sulle prestazioni previste per la ricostruzione
dei muoni su campioni simulati, scritta nel 2009 (Ref.~\Ref{mio:2009cscmuon}). \\
\newline
Dal punto di vista della analisi, si concentra sullo studio di eventi
$t\bar{t}$ con leptoni di grande impulso, selezionati con il {\it trigger}
di ATLAS. L'obiettivo \`e la misura della sezione d'urto di produzione
di coppie $t\bar{t}$ alla energia nel c.d.m. fornita da LHC, ed un confronto
con le predizioni del Modello Standard. Insieme con i colleghi di
Nikhef sta migliorando ed estendendo l'analisi a pi\`{u} livelli. Anzitutto \`e {\it main editor} della nota
sulle selezioni e la misura di efficienza per i leptoni in preparazione alla prima misura di sezione
d'urto con i dati di LHC. Inoltre \`e coordinatore
del gruppo di studi ed ottimizzazione della selezione di muoni per le analisi
di fisica del Top di ATLAS, all'interno del {\it Top Reconstruction} Working Group (Ref.~\Ref{AtlNotes:1});
inoltre lavora sul fit finale di {\it likelihood} al numero di candidati per la misura di sezione d'urto:
personalmente si occupa di migliorare la associazione di jets al top
che decade adronicamente, in modo da ridurre il principale fondo
(associazione errata) al fit al numero di candidati (Ref.~\Ref{AtlNotes:2}). \\
In particolare, con i primi dati intende investigare le risposte del detector in eventi complessi
come quelli di $t\bar{t}$, grazie alla conoscenza del Top ottenuta al
Tevatron. Sta anche lavorando con colleghi di NIKHEF alla determinazione, su campioni
simulati, della probabilita\`a attesa che un jet venga ricostruito
come un leptone di grande impulso e che induca alla selezione di un
falso evento $t\bar{t}$, al fine di parametrizzare tale probabilit\`a
nello spazio delle fasi.\\
\`E anche co-autore dell'{\tt Event Data Model} adottato dall'ATLAS Top
Reconstruction working group al fine di rappresentare univocamente i
candidati top nelle varie analisi (Ref.~\Ref{AtlNotes:3}).\\
\newline 
Nel 2007 \`{e} stato un {\it Research staff} alla University of
Washington, anche in questo caso distaccato al CERN. Ha attivamente
contribuito a studi di rivelabilit\`a di particella a vita media molto
lunga (alcuni metri), comuni in molti modelli di fisica oltre lo
SM. L'obiettivo \`e quello di sviluppare ed integrare in ATLAS nuovi
algoritmi di trigger sia di leptone che di jet {\it ad hoc} per massimizzare l'efficienza nella
selezione di vertici molto distanti dalla interazione primaria, pur
non saturando la banda passante con eventi di fondi quali eventi di
QCD a molti jet. Tale analisi necessita familiarit\`a con la struttura
del trigger di alto livello di ATLAS e una approfondita conoscenza
della risposta del detector e della ricostruzione in questi
particolari topologie. L'analisi \`e stata documentata in una nota di
ATLAS (Ref.~\Ref{AtlNotes:4}).
Lo studio \`e stato condotto in stretta collaborazione con colleghi di Roma.\\

Ha studiato fisica all'Universit\`{a} di Roma ``La Sapienza'', a
partire dal 1998. Ha ivi conseguito la laurea in fisica nel 2003 sotto
la supervisione del Professor Carlo Dionisi. Nel 2007 ottiene, presso
la medesima Universit\`{a}, il Dottorato di ricerca in fisica, con un
lavoro di tesi svolto nell'esperimento CDF. Si \`{e} recato con
regolarit\`{a} presso il Fermi National Accelerator Laboratory di
Chicago dal 2002 al 2006 per lavorare all'interno del gruppo di fisica
degli adroni $b$ di CDF: in particolare la sua tesi si situa
all'interno della recente misura della frequenza di {\it mixing}
$\Delta m_{s}$ dei mesoni $B_{s}^{0}$. Attesa per lungo tempo, questa misura rappresenta
un importante limite al contributo di Nuova Fisica nel settore della fisica dei
sapori pesanti ed \`{e} stata unicamente possibile al Tevatron di
Fermilab (Ref.~\Ref{Abulencia:2006obs}).\\ 
Il suo contributo inizia gi\`a con la tesi di laurea, per la quale
porta avanti studi di risoluzione e propone metodi di calibrazione {\it off-line} per il misuratore di Tempo
di Volo (TOF) di CDF utilizzando campioni puri come eventi
di-leptonici da decadimenti delle $J/\psi$, nei quali i due leptoni
vengono creati alle stesso tempo. Questo consente di separare effetti
di detector ed elettronica dalla parte algoritmica della misura di
tempo di volo (Ref.~\Ref{CDFNotes:18}). Il TOF fornisce a CDF la principale capacit\`a di separazione tra
Kaoni, pioni e protoni a basso momento, al fine di sviluppare
algoritmi di {\it flavour tagging} basato su mesoni $K$ per la analisi
di fenomeni dipendenti dal tempo, come il mixing. 
Egli stesso \`e l'autore del primo algoritmo funzionante di
{\it Opposite Side Kaon Tagging} ({\it flavour tagger} nel lato non di trigger utilizzando
mesoni $K$) ad un {\it collider} adronico (Ref.~\Ref{CDFNotes:2}).
Tale tagger, usato in
combinazione con gli altri {\it taggers} lontani dal lato di segnale
per la misura finale della frequenza di {\it mixing}, contribuisce ad
aumentare le prestazioni totali del flavour tagging (Ref.~\Ref{CDFNotes:1}).\\
Inoltre \`e responsabile di comparare le prestazioni del
TOF in vari periodi di presa dati, al fine di valutare le prestazioni
attese del {\it Same Side Kaon Tagger} nel tempo (Ref.~\Ref{CDFNotes:3}).
Il suo contributo sul TOF ed il flavour tagging sono \`e anche stato
successivamente rilevante per la misura della fase di mixing $\Delta
\Gamma_{s}$ (Ref.~\Ref{Aaltonen:2007dgs}) e per successive analisi utilizzanti 
particle identification;
alle quali egli contribuiva attivamente nel suo periodo di permanenza
a Berkeley.   
%\vbox{\vspace{0.5cm}}
\newpage
\begin{Istruzione}
Gennaio 2007 & Dottorato di ricerca in Fisica,
                 con una tesi dal titolo:
                 {\it ``First observation of $B_{s}$ mixing at the CDF II experiment with 
                    a newly developed Opposite Side $b$ flavour tagger using Kaons''},
                 sotto la supervisione del Prof. C. Dionisi e del Dott. M. Rescigno,
                 Universit\`a degli Studi di Roma {\em La Sapienza}, Roma, Italia \\ \\

Settembre 2003 & Laurea in Fisica con la votazione di 110/110,
                 con una tesi dal titolo:
                 {\it ``Studio della risoluzione del rivelatore di
                   Tempo di Volo dell'esperimento CDF II al Fermilab''},  
                 sotto la supervisione del Prof. C. Dionisi e del Dott. S. Giagu, 
                 Universit\`a degli Studi di Roma {\em La Sapienza}, Roma, Italia \\ \\

Luglio 1998 & Maturit\`a Classica con votazione 60/60,\\
          & Liceo Ginnasio Pilo Albertelli, Roma, Italia\\
\end{Istruzione}

\begin{Carriera}
2008 -- attuale             & Post-doctoral staff a NIKHEF, distaccato al CERN \\
2008 -- attuale             & Membro della collaborazione ATLAS come affiliato a NIKHEF \\
2007                          & Research Associate alla University of Washington, distaccato al CERN \\
2006                          & {\it Visiting scientist} al Lawrence
Berkeley National Laboratory per collaborare con il locale gruppo di
CDF sulle analisi di $B_{s}$ mixing  \\
2002 -- 2006               & Regolari missioni al Fermi National
Accelerator Laboratory per l'esperimento CDF \\
2003 -- 2006               & Studente di dottorato, con il gruppo CDF
presso l'Universit\`a degli Studi di Roma {\em La Sapienza} e sezione
INFN di Roma\\
2002 -- 2006               & Membro della collaborazione CDF come affiliato a INFN-Roma \\
2003 -- 2006               & Associazione INFN per l'esperimento CDF\\
2004 -- 2005               & Scelto come aiuto alla didattica presso la Universit\`a di Roma {\em La Sapienza} \\
\end{Carriera}

\begin{Borse di studio e ricerca}
2003 -- 2006               & Borsa di studio INFN per il Dottorato di
ricerca, assegnata per graduatoria del concorso di accesso al
Dottorato di Ricerca \\
\end{Borse di studio e ricerca}

\begin{Scuole}
2004 & CERN European Summer School, Sant Feliu de Guixols, Spagna\\
\end{Scuole}

\begin{Insegnamento}
2010 & Supervisione della studentessa N.Ruckstuhl (NIKHEF) nei suoi studi di {\it
  scala e risoluzione sulla misura di impulso dei muoni} con raggi cosmici e i primi dati di LHC \\
2009 & Supervisione dello studente E.J.Schioppa (INFN-Roma) nel
suo studio di {\it timing del trigger di muoni a Livello-1} con raggi cosmici, 
utilizzando un trigger esterno, durante il periodo
come {\it summer student} al CERN \\
2008 & Supervisione dello studente A.Doxiadis (NIKHEF) nei suoi studi di {\it
  fake leptons} per misure di $t\bar{t}$ \\
2007 & Supervisione dello studente D.Ventura (Univerity of Washington) \\
2005 & Supervisione dello studente M.Nardecchia (INFN-Roma) nel
suo studio di {\it flavour tagging} con barioni $\Lambda$, durante il periodo
come {\it summer student} al Fermilab \\
2004 -- 2005 & Aiuto alla didattica per il corso di Fisica, Corso di
laurea in Farmacia, Universit\`a di Roma {\em La Sapienza}\\
\end{Insegnamento}
\newpage
\begin{Talks a conferenze e Seminari}
Feb 2010 & ``Measurement of the Top quark pair production at ATLAS with the first data from LHC'' Seminario di Laboratorio CPPM, Centre de Physique de Particules de Marseille, Marseille, France \\
Ott 2009 & ``Results from the ATLAS Barrel Level-1 Muon Trigger timing studies using combined trigger and offline tracking'' 2009 IEEE Nuclear Science Symposium and Medical Imaging Conference (IEEE NSS MIC 09), Orlando, FL, USA \\
Gen 2009 & ``Early Top physics with ATLAS at the LHC'' Physics@FOM Veldhoven 2009 \\
Lug 2006 & ``Measurement of $B_{s}$ oscillations at CDF'' 7th International Conference on Hyperons, Charm And Beauty Hadrons (BEACH06),
             Lancaster, UK. \\
Apr 2006 & ``Measurement of $B_{s}$ oscillation frequency at CDF''
            Incontri di Fisica delle Alte Energie, Pavia, Italy,\\
Feb 2006 & ``$B_{s}$ and sensitivity to new physics at CDF''
            Third workshop on $b$ physics, Parma, Italy,\\
Lug 2005 & ``Techniques for $B_{s}$ Mixing at CDF'' 
           poster at the Hadron Collider Physics Symposium 2005, Les Diablerets, Switzerland\\
Apr 2005 & ``Opposite side B-flavour tagging using combined TOF and dE/dx particle identification technique''
           American Physics Society April Meeting 2005, Tampa, FL, USA \\
Feb 2006 & ``$b$ flavour tagging with Kaons for B physics at CDF''
           RTN ``The third generation as a probe for new physics'' Meeting, Karlsruhe, Germany\\
\end{Talks a conferenze e Seminari}

\begin{Conoscenze tecniche}
\item {\bf Programmazione}: C, C++, programma di analisi dati {\tt Root}, codice di analisi dati di ATLAS ATHENA.
      Buona conoscenza di Linux e Windows sia a livello utente che amministratore.
\item {\bf Computer Hardware}: PC incluse periferiche ed interfacce
\item {\bf Rivelatori di particelle}: operation, calibrazioni, trigger e sincronizzazione della elettronica di trigger
\end{Conoscenze tecniche}

%\begin{References}
%\vbox{\vspace{5mm}}           \\ 
%Prof. Carlo Dionisi                               \\
%Dipartimento di Fisica                          \\
%Universit\`a degli Studi di Roma ``La Sapienza''   \\
%Piazzale Aldo Moro, 2                            \\
%Rome, Italy 00186                               \\
%+39 (06) 4991-4328 Voice                        \\
%+39 (06) 4454-835 Fax                           \\
%{\tt Carlo.Dionisi@roma1.infn.it}               \\
%\vbox{\vspace{20mm}}                             \\                                                
%Prof. Marjorie Dale Shapiro      \\
%Physics Division                \\
%Lawrence Berkeley National Laboratory       \\
%1 Cyclotron Road                 \\
%Berkeley, CA 94720 USA       \\
%+1 (510) 486-4683 Voice      \\
%{\tt mdshapiro@lbl.gov}       \\
%\vbox{\vspace{20mm}}           \\ 
%Prof. Kevin T.Pitts                                      \\
%Department of Physics                                    \\
%University of Illinois at Urbana-Champaign              \\
%1110 West Green Street                                   \\
%Urbana, IL 61801-3080 USA                                \\
%+1 (217) 333-3946 Voice                                 \\
%+1 (217) 333-4990 Fax                                    \\
%+1 (630) 840-8718 Fermilab Office                        \\
%{\tt kpitts@uiuc.edu}                                    \\
%\end{References}

\SetAttitionalText{Lista delle pubblicazioni nelle quali il proprio contributo \`e stato fondamentale. Gli articoli indicati con
  ~\RefereedFlag ~ sono stati sottoposti al giudizio di un {\it referee}} 
\begin{Selezione delle Pubblicazioni}
\EnableMostRelevant
\MostRelevant
\DisableRefereed
\FlagRefereed
\input{../PUBBLICATION_LIST/publist_giuseppe}
\end{Selezione delle Pubblicazioni}

\clearpage

\SetAttitionalText{Lista delle note di CDF ed ATLAS delle quali \`e
  uno degli autori}
\begin{Note interne e pubbliche}
\DisableMostRelevant
\DisableRefereed
%----------------------------------------------------------------------------------------------
\ChooseFlag{\IsNotRefereed\IsMostRelevant}{
\alexcvlabel{AtlNotes:20121}
    \alexcvtitle     {Measurement of the top quark pair production cross section with ATLAS in pp collisions at $\sqrt{s}$ = 7 TeV in the single-lepton channel using semileptonic $b$ decays} 
    \alexcvatlconfnote      {ATL-CONF-2012-131}
}

\ChooseFlag{\IsNotRefereed\IsNotMostRelevant}{
\alexcvlabel{AtlNotes:01}
    \alexcvtitle     {Measurement of the Top Quark Pair Production Cross-section in ATLAS in the Single Lepton plus Jets Channel}
    \alexcvatlnote      {ATL-COM-PHYS-2011-666}
}

\ChooseFlag{\IsNotRefereed\IsMostRelevant}{
\alexcvlabel{AtlNotes:03}
    \alexcvtitle     {Measurement of the top quark-pair cross-section with ATLAS in pp collisions at sqrt(s) = 7 TeV in the single-lepton channel using b-tagging}
    \alexcvatlnote      {ATL-COM-CONF-2011-028}
}


\ChooseFlag{\IsNotRefereed\IsNotMostRelevant}{
\alexcvlabel{AtlNotes:04}
    \alexcvtitle     {Di-muon invariant mass resolution in Z->mu+mu- decays : document supporting the request for approval of performance plots}
    \alexcvatlnote      {ATL-COM-CONF-2010-904}
}

\ChooseFlag{\IsNotRefereed\IsMostRelevant}{
\alexcvlabel{AtlNotes:05}
    \alexcvtitle     {Calibration of the $\chi^{2}_{match}$-based Soft Muon Tagger algorithm}
    \alexcvatlnote      {ATL-COM-PHYS-2012-008}
}

\ChooseFlag{\IsNotRefereed\IsMostRelevant}{
\alexcvlabel{AtlNotes:06}
    \alexcvtitle     {Calibration of the $\chi^{2}_{match}$-based Soft Muon Tagger algorithm using 2012 ATLAS data}
    \alexcvatlnote      {ATL-COM-MUON-2013-031}
}

\ChooseFlag{\IsNotRefereed\IsMostRelevant}{
\alexcvlabel{AtlNotes:07}
    \alexcvtitle     {Object Selections and Background estimates in the $H\rightarrow WW^{(*)}$ analysis with 20.7 $fb^{-1}$ of data collected with the ATLAS detector at $\sqrt{s} = 8$ TeV}
    \alexcvatlnote      {ATL-COM-PHYS-2013-1504}
}

\ChooseFlag{\IsNotRefereed\IsNotMostRelevant}{
\alexcvlabel{AtlNotes:1}
    \alexcvtitle     {Study on reconstructed object definition and selection for top physics}
    \alexcvatlnote      {ATL-COM-PHYS-2009-633}
}

\ChooseFlag{\IsNotRefereed\IsMostRelevant}{
\alexcvlabel{AtlNotes:20101}
    \alexcvtitle     {Performance of the ATLAS Muon Trigger in p-p collisions at $\sqrt{s}$ = 7 TeV}
    \alexcvatlconfnote      {ATL-CONF-2010-095}
}

\ChooseFlag{\IsNotRefereed\IsNotMostRelevant}{
\alexcvlabel{AtlNotes:02}
    \alexcvtitle     {ATLAS muon reconstruction efficiency and dimuon mass resolution in early 2011 LHC collisions data}
    \alexcvatlnote      {ATL-COM-CONF-2011-094}
}


\ChooseFlag{\IsNotRefereed\IsMostRelevant}{
\alexcvlabel{AtlNotes:55}
    \alexcvtitle     {{\bf Main editor:} Object selection and calibration, background estimations and MC samples for the Winter 2012 Top Quark analyses with 2011 data}
    \alexcvatlnote      {ATL-COM-PHYS-2012-224, ATL-COM-PHYS-2012-449 (update for Summer 2012), ATL-COM-PHYS-2013-088 (update for Winter 2013)}
}

\ChooseFlag{\IsNotRefereed\IsNotMostRelevant}{
\alexcvlabel{AtlNotes:2}
   \alexcvtitle     {Prospects for measuring the Top Quark Pair Production Cross-section in the Single Lepton Channel at ATLAS in 10 TeV $p-p$ Collisions}
   \alexcvatlpubnote      {ATL-PHYS-INT-2009-071, ATL-COM-PHYS-2009-306}
}

\ChooseFlag{\IsNotRefereed\IsMostRelevant}{
\alexcvlabel{AtlNotes:4}
    \alexcvtitle     {Detection of long lived neutral particles in the ATLAS detector}
    \alexcvatlpubnote     {ATL-COM-PHYS-2008-020}
}

\ChooseFlag{\IsNotRefereed\IsMostRelevant}{
\alexcvlabel{AtlNotes:8}
    \alexcvtitle     {Muon reconstruction performance}
    \alexcvatlpubnote      {ATLAS-CONF-2010-064}
}

\ChooseFlag{\IsNotRefereed\IsNotMostRelevant}{
\alexcvlabel{AtlNotes:9}
    \alexcvtitle     {Search for top pair candidate events in ATLAS at $\sqrt{s}~=~7~ TeV$}
    \alexcvatlconfnote      {ATLAS-CONF-2010-063}
}

\ChooseFlag{\IsNotRefereed\IsMostRelevant}{
\alexcvlabel{AtlNotes:12}
      \alexcvtitle     {{\bf Main editor:} Lepton Trigger and Identification for the first {\it Top quark} observation
}
\alexcvatlnote      {ATLAS-COM-PHYS-2010-826}
}

\ChooseFlag{\IsNotRefereed\IsNotMostRelevant}{
\alexcvlabel{AtlNotes:14}
      \alexcvtitle     {{\bf Main editor:} Muon momentum scale and resolution measurements with inclusive muons from $1.2~pb^{-1}$ of collision data at $\sqrt{s}$ = 7 TeV
}
\alexcvatlnote      {ATL-COM-PHYS-2010-708}
}

\ChooseFlag{\IsNotRefereed\IsNotMostRelevant}{
\alexcvlabel{AtlNotes:3}
   \alexcvtitle     {{\bf Main editor:} Design Considerations for the Top reconstruction Output EDM Classes}
   \alexcvatlnote      {ATL-COM-SOFT-2009-006}
}

\ChooseFlag{\IsNotRefereed\IsNotMostRelevant}{
\alexcvlabel{AtlNotes:5}
    \alexcvtitle     {Accompanying note for approval of plots from Level 1 Muon Barrel trigger timing studies}
    \alexcvatlnote      {ATL-COM-MUON-2009-034}
}

\ChooseFlag{\IsNotRefereed\IsNotMostRelevant}{
\alexcvlabel{AtlNotes:6}
    \alexcvtitle     {Atlas Muon Trigger Performance on cosmics and p-p collisions at $\sqrt{s}~=~900~GeV$}
    \alexcvatlpubnote    {ATL-COM-DAQ-2010-011}
}

\ChooseFlag{\IsNotRefereed\IsNotMostRelevant}{
\alexcvlabel{AtlNotes:7}
    \alexcvtitle     {Muon Performance in Minimum Bias $pp$ Collision Data at $\sqrt{s}~=~7~TeV$ with ATLAS}
    \alexcvatlconfnote      {ATLAS-CONF-2010-036}
}

\ChooseFlag{\IsNotRefereed\IsNotMostRelevant}{
\alexcvlabel{AtlNotes:10}
    \alexcvtitle     {Expected event distributions for early top pair candidates in ATLAS at $\sqrt{s}~=~7~ TeV$}
    \alexcvatlpubnote      {ATL-PHYS-PUB-2010-012}
}

\ChooseFlag{\IsNotRefereed\IsNotMostRelevant}{
\alexcvlabel{AtlNotes:11}
    \alexcvtitle     {Measurement of the $W \rightarrow \ell\nu$ production cross-section and observation of $Z \rightarrow \ell\ell$ production in proton-proton collisions at $\sqrt{s}~=~7~TeV$ with the ATLAS detector}
    \alexcvatlconfnote      {ATL-CONF-2010-051}
}

\ChooseFlag{\IsNotRefereed\IsMostRelevant}{
\alexcvlabel{AtlNotes:13}
      \alexcvtitle     {Background studies for top-pair production in lepton plus jets final states in $\sqrt{s}~=~7~TeV$ ATLAS data
}
\alexcvatlnote      {ATLAS-COM-CONF-2010-085}

}

\ChooseFlag{\IsNotRefereed\IsMostRelevant}{
\alexcvlabel{AtlNotes:14}
      \alexcvtitle     {{\bf Main editor:} Muon Momentum Resolution in First Pass Reconstruction of pp Collision Data Recorded by ATLAS in 2010
}
\alexcvatlconfnote      {ATLAS-CONF-2011-046}
}

\ChooseFlag{\IsNotRefereed\IsMostRelevant}{
\alexcvlabel{AtlNotes:15}
      \alexcvtitle     {Measurement of the top quark-pair cross-section with ATLAS in pp collisions at $\sqrt{s}~=~7~TeV$ in the single-lepton channel using b-tagging}
\alexcvatlconfnote      {ATLAS-CONF-2011-035}
}

\ChooseFlag{\IsNotRefereed\IsMostRelevant}{
\alexcvlabel{AtlNotes:16}
      \alexcvtitle     {{\bf Main editor:} Lepton trigger and identification for the Winter 2011 top quark analyses}
\alexcvatlnote      {ATL-COM-PHYS-2011-123}
}

\ChooseFlag{\IsNotRefereed\IsNotMostRelevant}{
\alexcvlabel{AtlNotes:17}
      \alexcvtitle     {Measurement of the top quark cross-section in the semileptonic channel at $sqrt{s}=7TeV$ with the ATLAS detector}
\alexcvatlnote      {ATL-COM-PHYS-2011-111}
}

\ChooseFlag{\IsNotRefereed\IsNotMostRelevant}{
\alexcvlabel{CDFNotes:1}
     \alexcvtitle     {Determination of $B^{0}$ and $B^{+}$ Lifetimes
     in Hadronic Decays Using Partially and Fully Reconstructed Modes
     without Event-by-Event $ct$ Resolutions}
     \alexcvcdfnote      {9139}  
}

\ChooseFlag{\IsNotRefereed\IsNotMostRelevant}{
\alexcvlabel{CDFNotes:2}
     \alexcvtitle     {First observation of $\bar{B}_{s}^{0} \rightarrow
    D_{s}^{\pm}K^{\mp}$ and measurement of the relative branching
     fraction BR($\bar{B}_{s}^{0} \rightarrow
     D_{s}^{\pm}K^{\mp}$)/BR($\bar{B}_{s}^{0} \rightarrow
     D_{s}^{+}\pi^{-}$)}
    \alexcvcdfnote      {8850}
}
\ChooseFlag{\IsNotRefereed\IsNotMostRelevant}{
\alexcvlabel{CDFNotes:3}
     \alexcvtitle     {Determination of $B^{0}$ and $B^{+}$ Lifetimes
     in Hadronic Decays Using Partially and Fully Reconstructed Modes}
     \alexcvcdfnote      {8778}
}
\ChooseFlag{\IsNotRefereed\IsNotMostRelevant}{
\alexcvlabel{CDFNotes:4}
     \alexcvtitle     {Measurement of $B^{-}$ Relative Branching
     Fractions with a Combined Mass and $dE/dx$ Fit}
     \alexcvcdfnote      {8777}
}
\ChooseFlag{\IsNotRefereed\IsNotMostRelevant}{
\alexcvlabel{CDFNotes:5}
     \alexcvtitle     {Measurement of $B_{s}^{0}$ Branching
     Fractions Using Combined Mass and $dE/dx$ Fits}
     \alexcvcdfnote      {8776}
}
\ChooseFlag{\IsNotRefereed\IsNotMostRelevant}{
\alexcvlabel{CDFNotes:6}
     \alexcvtitle     {Measurement of $B^{0}$ Branching
     Fractions Using Combined Mass and $dE/dx$ Fits}
     \alexcvcdfnote      {8705}
}
\ChooseFlag{\IsNotRefereed\IsNotMostRelevant}{
\alexcvlabel{CDFNotes:7}
     \alexcvtitle     {$B_{s} \rightarrow
     D_{s}\pi$ with $D_{s} \rightarrow
     K_{s}[\pi\pi]K$ channel reconstruction using PID}
     \alexcvcdfnote      {8520}
}

\ChooseFlag{\IsNotRefereed\IsMostRelevant}{
\alexcvlabel{CDFNotes:8}
     \alexcvtitle     {Combined opposite side flavor tagger}
     \alexcvcdfnote      {8314}
}

\ChooseFlag{\IsNotRefereed\IsMostRelevant}{
\alexcvlabel{CDFNotes:9}
     \alexcvtitle     {Opposite Side Kaon Tagging}
     \alexcvcdfnote      {8179}
}
\ChooseFlag{\IsNotRefereed\IsMostRelevant}{
\alexcvlabel{CDFNotes:10}
     \alexcvtitle     {dE/dx, TOF validation studies on 0h/0i data for Bs mixing analyses}
     \alexcvcdfnote      {8169}
}

\ChooseFlag{\IsNotRefereed\IsMostRelevant}{
\alexcvlabel{CDFNotes:18}
     \alexcvtitle     {Tof Resolution studies using muons from $J/\psi$}
     \alexcvcdfnote      {6810}
}

\ChooseFlag{\IsNotRefereed\IsNotMostRelevant}{
\alexcvlabel{CDFNotes:11}
     \alexcvtitle     {Improving the dE/dx modeling and a study of the composition of the $B \rightarrow hh$ background}
     \alexcvcdfnote      {7646}
}
\ChooseFlag{\IsNotRefereed\IsMostRelevant}{
\alexcvlabel{CDFNotes:12}
     \alexcvtitle     {Particle Identification by combining TOF and dE/dx information}
     \alexcvcdfnote      {7488}
}
\ChooseFlag{\IsNotRefereed\IsNotMostRelevant}{
\alexcvlabel{CDFNotes:13}
     \alexcvtitle     {$B\_PIPI$ trigger at high luminosity}
     \alexcvcdfnote      {7320}
}
\ChooseFlag{\IsNotRefereed\IsNotMostRelevant}{
\alexcvlabel{CDFNotes:14}
     \alexcvtitle     {Measurement of isolation efficiency in low pt B mesons}
     \alexcvcdfnote      {7066}
}
\ChooseFlag{\IsNotRefereed\IsNotMostRelevant}{
\alexcvlabel{CDFNotes:15}
     \alexcvtitle     {Branching Ratios and CP asymmetries in $B \rightarrow hh$ decays from 180 $pb^{-1}$}
     \alexcvcdfnote      {7049}
}

\ChooseFlag{\IsNotRefereed\IsNotMostRelevant}{
\alexcvlabel{CDFNotes:16}
     \alexcvtitle     {Upper limit on $\Lambda_b$ to hh}
     \alexcvcdfnote      {7048}
}
\ChooseFlag{\IsNotRefereed\IsNotMostRelevant}{
\alexcvlabel{CDFNotes:17}
     \alexcvtitle     {Search for $B_{s} \rightarrow \phi \phi$ decays at CDF}
     \alexcvcdfnote      {6937}
}


\end{Note interne e pubbliche}


\clearpage
%
\SetAttitionalText{Lista di tutti gli articoli pubblicati dei quali \`e
  uno degli autori}
\begin{Pubblicazioni sotto referee}
% Pick only Refereed papers
\EnableRefereed
\Refereed
% without flagging them
%\DontFlagRefereed
% and removing objects from the ``most relevant'' list
\EnableMostRelevant
\NotMostRelevant
\input{../PUBBLICATION_LIST/publist_giuseppe}
\end{Pubblicazioni sotto referee}


%\vbox{\vspace{20cm}}

\end{vita}

%-------------------------------------------------------------------------------

%\section*{Research Plans}
%\input{research_interest_atlas.tex}

%\section*{Teaching Activity}

\end{document}
